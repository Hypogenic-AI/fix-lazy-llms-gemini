Our experiments reveal that while simply forcing the model to work ``harder'' (longer) does not guarantee better results, combining motivation (persona) with constraints (budget) yields significant improvements.

\subsection{Main Results}

\para{Reasoning (GSM8K)}.
As shown in \cref{tab:gsm8k}, the \textbf{Combined} strategy achieves the highest accuracy of \textbf{90.0\%}, a 4.0\% improvement over the Baseline. Interestingly, neither intervention worked well in isolation. The \textbf{Harsh Critic} persona alone significantly degraded performance (-6.0\%), suggesting that the adversarial tone may distract the model from the logical task. Similarly, \textbf{Budget Control} alone reduced accuracy (-4.0\%), indicating that forcing length without a quality focus may induce ``verbose hallucinations.'' The synergy of the Combined approach implies that the Critic provides the necessary focus for the extra Budget.

\begin{table}[t]
    \centering
    \caption{Accuracy on GSM8K (Reasoning) and TruthfulQA (Factuality). Best results in \textbf{bold}.}
    \label{tab:gsm8k}
    \resizebox{0.9\textwidth}{!}{
    \begin{tabular}{@{}llcc@{}}
        \toprule
        \textbf{Condition} & \textbf{Mechanism} & \textbf{GSM8K Acc} & \textbf{TruthfulQA Acc} \\
        \midrule
        Baseline & None & 86.0\% & 82.0\% \\
        Harsh Critic & Subjective Pressure & 80.0\% \decrease & \textbf{88.0\%} \increase \\
        Budget Control & Objective Constraint & 82.0\% \decrease & 78.0\% \decrease \\
        \textbf{Combined} & Pressure + Constraint & \textbf{90.0\%} \increase & \textbf{88.0\%} \increase \\
        \bottomrule
    \end{tabular}
    }
\end{table}

\para{Factuality (TruthfulQA)}.
For knowledge tasks, the dynamic shifts. The \textbf{Harsh Critic} persona alone proved highly effective, improving accuracy by 6.0\% to \textbf{88.0\%}. This aligns with the nature of the task: detecting misconceptions requires a skeptical stance, which the critical persona naturally enforces. Budget Control alone was detrimental (-4.0\%), again suggesting that unstructured verbosity is not a proxy for truthfulness.

\subsection{The Role of Persona: Rude vs. Skeptical}

We investigated whether the ``rudeness'' of the Harsh Critic was the active ingredient. We compared it against a ``Polite High Standards'' persona and a ``Skeptical Scientist'' persona (see \cref{fig:persona}).

\begin{figure}[t]
    \centering
    \includegraphics[width=0.7\linewidth]{figures/persona_comparison.png}
    \caption{Impact of Persona Tone. While the ``Harsh Critic'' helps factuality, the ``Skeptical Scientist'' achieves optimal performance across both domains.}
    \label{fig:persona}
\end{figure}

The \textbf{Skeptical Scientist} persona emerged as the superior strategy, achieving \textbf{90\% accuracy} on TruthfulQA (outperforming Harsh Critic) and maintaining baseline-level reasoning on GSM8K (86\%). This suggests that \textit{rigor}, not hostility, is the key driver of performance. The ``Polite'' persona failed to improve TruthfulQA (80\%), indicating that agreeableness---even with high standards---is a liability for truth-seeking.

\subsection{Efficiency Analysis}

We analyzed the cost of these improvements in terms of response length (see \cref{fig:efficiency}). The \textbf{Combined} strategy comes at a high cost, increasing response length by $\sim$50\% (246.9 words vs 163.8 baseline) to achieve its 4\% gain. In contrast, the \textbf{Skeptical Scientist} is the most efficient strategy (\cref{fig:efficiency}), improving accuracy without significantly inflating the token count.

\begin{figure}[t]
    \centering
    \begin{subfigure}[b]{0.48\textwidth}
        \includegraphics[width=\textwidth]{figures/efficiency_plot.png}
        \caption{Efficiency Trade-off}
        \label{fig:efficiency}
    \end{subfigure}
    \begin{subfigure}[b]{0.48\textwidth}
        \includegraphics[width=\textwidth]{figures/response_lengths.png}
        \caption{Response Lengths}
        \label{fig:lengths}
    \end{subfigure}
    \caption{Efficiency analysis. The ``Skeptical Scientist'' (bottom right in \subref{fig:efficiency}) represents the optimal trade-off between accuracy and token cost.}
    \label{fig:eff_len}
\end{figure}
